\documentclass[10pt,a4paper]{article}
\usepackage[utf8]{inputenc}
\usepackage{amsmath}
\usepackage{amsfonts}
\usepackage{amssymb}
\usepackage{graphicx}
\usepackage{enumerate}
\begin{document}

\section{Set Theory}

To demystify mathematics consider
\begin{enumerate}[(i)]
\item What is a theorem?
\item What is a proof?
\end{enumerate}
What if we don't know the answer?

To begin we need
\begin{enumerate}[(a)]
\item an example(s)
\item a nearly related concept
\end{enumerate}


\includegraphics[scale=.5]{Pages/ST_3}

\newpage

Related Concept: Greek Syllogism

\underline{example:}
\begin{enumerate}
\item All men are mortal.
\item Socrates is a man.
\item Therefore, Socrates must die. 
\end{enumerate}

To analyze, recast in set theoretic terms via Venn Diagram.

\includegraphics[scale=.5]{Pages/ST_4}

\newpage

\includegraphics[scale=.2]{Pages/ST_5_im1}

$S$: Socrates\\
$M$: Set of Men\\
$D$: Things that will die\\
$\mathcal{U}$: Things on Earth

\includegraphics[scale=.5]{Pages/ST_5} 



%Zack: Pages 6,7,8,19,20

%Jack: 21, 9, 10, 11

%Koka: Pages 13, 13A, 22 ,22A, 22B

\includegraphics[scale=.5]{Pages/Page_13}

13. Suppose we require that these 3 operations $^C , \bigcup , \bigcap$ always produce sets. Then $A \bigcap A^C \equiv \{ x \in u:x \in A$ and $x \notin $ must be a set in U. 
It is called \underline{the empty set} $\emptyset$. The set with no elements. 

\vspace{.20 in}

\includegraphics[scale=.5]{Pages/Page_13A}

13A. 
\begin{enumerate}
\item $$B \bigcap (\bigcup_{\alpha \in J} A_\alpha) \stackrel{?}{=}$$
\item $$B \bigcup (\bigcap_{\alpha \in J} A_\alpha) \stackrel{?}{=}$$
\item $$(\bigcap_{\alpha \in J} A_\alpha)^c \stackrel{?}{=}$$
\item $$(\bigcup_{\alpha \in J} A_\alpha)^c \stackrel{?}{=}$$
\end{enumerate}

\vspace{.20 in}

\includegraphics[scale=.5]{Pages/Page_22}

22. What do we do next? Begin constructing sets, $\{$ apple $\}$ , $\{$ pen $\}$, $\{$ grape $\}$, $\{$ eagle $\}$, $\{$ bear, deer $\}$, $\{$ pencil, paper $\}$, $\{$ sun, moon, onion. $\}$ These examples motivate the need to

\vspace{.20 in}

\includegraphics[scale=.5]{Pages/Page_22A}

22A. What next? Suppose we try to generate all sets from simple to complex Examples $\{$ apple $\}$  $\{$ pear $\}$ $\{$ cat $\}$ $\{$ dog,hat $\}$ $\{$ atoms in your body $\}$ What makes these sets most similar? 

\vspace{.20 in}

\includegraphics[scale=.5]{Pages/Page_22B}

22B. Notice $\{$ apple $\}$ and $\{$ pear $\}$ are merely relating of one another as one $\{$ dog, hat $\}$ and $\{$ cat, oval $\}$ What is a relabeling of a set? Perhaps sets A and B have the same 'size' if they are relabeling of one another. We seem to be dancing around the concept of number. 

\section{Generate $\mathbb{N}$}


%Ruth: Pages L4A-L4G




\section{From $\mathbb{Z}$ to $\mathbb{R}$ via ordering}
%Jazz: ZR1-ZR5

%Kyler: ZR6 - ZR10

%Preethika: ZR11-ZR14


\section{Sequence and Limits}

%Aaron: First 2 pages and 48-50

%Hamza: 51-52B

\section{Limit and Convergence}

%Joe: 50-51

%Quinten: 52-53

%Farishta: 53A-54A

\section{Infinite Series}

%Sukhreet: IS1 - IS 7

%Matthew: IS8 - IS15

%Will: IS16 - IS23

%Rebecca: IS24 - IS32

%Maady: IS33 - IS42

\section{Metric Spaces Part 1}

%Travis: M1 - M5

%Jerome: M6- M10



\section{Metric Spaces Part 2}


%Bryant: M1-M7

%Reshma: M8-M14

%Ethan: M15-M21





\end{document}