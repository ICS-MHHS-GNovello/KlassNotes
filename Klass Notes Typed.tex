\documentclass[10pt,a4paper]{article}
\usepackage[utf8]{inputenc}
\usepackage{amsmath}
\usepackage{amsfonts}
\usepackage{amssymb}
\usepackage{graphicx}
\usepackage{enumerate}
\begin{document}

\section{Set Theory}

To demystify mathematics consider
\begin{enumerate}[(i)]
\item What is a theorem?
\item What is a proof?
\end{enumerate}
What if we don't know the answer?

To begin we need
\begin{enumerate}[(a)]
\item an example(s)
\item a nearly related concept
\end{enumerate}


\includegraphics[scale=.5]{Pages/ST_3}

\newpage

Related Concept: Greek Syllogism

\underline{example:}
\begin{enumerate}
\item All men are mortal.
\item Socrates is a man.
\item Therefore, Socrates must die. 
\end{enumerate}

To analyze, recast in set theoretic terms via Venn Diagram.

\includegraphics[scale=.5]{Pages/ST_4}

\newpage

\includegraphics[scale=.2]{Pages/ST_5_im1}

$S$: Socrates\\
$M$: Set of Men\\
$D$: Things that will die\\
$\mathcal{U}$: Things on Earth

\includegraphics[scale=.5]{Pages/ST_5} 



%Zack: Pages 6,7,8,19,20

%Jack: 21, 9, 10, 11

%Koka: Pages 13, 13A, 22 ,22A, 22B


\section{Generate $\mathbb{N}$}


%Ruth: Pages L4A-L4G




\section{From $\mathbb{Z}$ to $\mathbb{R}$ via ordering}
%Jazz: ZR1-ZR5

%Kyler: ZR6 - ZR10

%Preethika: ZR11-ZR14


\section{Sequence and Limits}

%Aaron: First 2 pages and 48-50

%Hamza: 51-52B

\section{Limit and Convergence}

%Joe: 50-51

%Quinten: 52-53

%Farishta: 53A-54A

\section{Infinite Series}

%Sukhreet: IS1 - IS 7

%Matthew: IS8 - IS15

%Will: IS16 - IS23

%Rebecca: IS24 - IS32

%Maady: IS33 - IS42
\newpage 

\huge \underline{Corollary}

\large(estimating series)

\normalsize Let $b_1 \geq b_2 \geq$ ...,  $b_n \rightarrow 0$ 

then $$\sum_{j=1}^{\infty}(-1)^{j-1} b_j $$converges 

\underline{Proof:} $$A_n = \sum^n_{J=1}(-1)^{j-1}$$ is less 

Hence $$s_n = \sum^n_{J=1}(-1)^{j-1}b_j$$

converges in $R$,


$$0 \leq s_2 \leq s_4 \leq ..,$$

$$so$$ 

\includegraphics[scale=.42]{Pages/IS_33}


\newpage

$\exists d \leq s_\infty \le \infty$ such that

$s_n \rightarrow s_\infty$, moreover, 

$*s_{2n} \leq s_\infty \leq s_{2n-1}$

since $s_1 \geq s_3 \geq$ ...

consequently 

$|s_\infty - s_n| \leq b_{n+1}$

\includegraphics[scale=.42]{Pages/IS_34}


\newpage

\huge \underline{Power Series}

\normalsize let $$f(x) = \sum^{\infty}_{j=0}a_jx^j$$

Then $\exists 0\leq R \leq \infty$  such that

$$\sum^{\infty}_{j=0}a_jx^j$$ converges absolutely for $0 \leq |x| < R.$ and diverges for $|x|>R$

\includegraphics[scale=.42]{Pages/IS_35}

\newpage 

Proof:Want 
$|a_jx^j| \leq \lambda^j <1$ for converges 

So, taking $j^{th}$ roots,
need $|x| |a_j|^{\frac{1}{j}} \leq \lambda <1$ 

Need $$|x| \limsup_{j\rightarrow\infty} |a_j|^{\frac{1}{j}} <1$$

$$|x| < \frac{1}{\bar{\lim}_{j \rightarrow \infty} |a_j|^{\frac{1}{j}}} \equiv R$$

Proof $$|x|>R, \bar{\lim}_{J\rightarrow\infty}|x^ja_j|=\infty$$

\includegraphics[scale=.42]{Pages/IS_36}
 

\section{Metric Spaces Part 1}

%Travis: M1 - M5

%Jerome: M6- M10



\section{Metric Spaces Part 2}


%Bryant: M1-M7

%Reshma: M8-M14

%Ethan: M15-M21





\end{document}